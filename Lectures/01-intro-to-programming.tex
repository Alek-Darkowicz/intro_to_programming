\documentclass[aspectratio=169]{beamer}

\usepackage[utf8]{inputenc}
\usepackage{textcomp}
\usepackage[official]{eurosym}
\usepackage[polish]{babel}
\usepackage{amsthm}
\usepackage{graphicx}
\usepackage[T1]{fontenc}
\usepackage{scrextend}
\usepackage{hyperref}
\usepackage{xcolor}
\usepackage{geometry}
\usepackage{listings}
\usepackage{epigraph}

\usetheme{-bjeldbak/beamerthemebjeldbak}

\definecolor{xbarcolor}{HTML}{0f2f0f}
\setbeamercolor{lower separation line head}{bg=xbarcolor} 
\setbeamercolor{lower separation line foot}{bg=xbarcolor} 

\title{Podstawy programownia (w języku C++)}
\subtitle{Wstęp do programowania}
\author{Marek Marecki}
\institute{Polsko-Japońska Akademia Technik Komputerowych}

\lstset{basicstyle=\ttfamily\color{black},
columns=fixed,
escapeinside={\%*}{*)},
inputencoding=utf8,
extendedchars=true,
moredelim=**[is][\color{red}]{@}{@}}

\begin{document}

{%
    \setbeamertemplate{headline}{}
    \frame{\titlepage}
}

\section{Rys historyczny}

\begin{frame}
    \frametitle{Środek historii}
    \framesubtitle{...czyli 150 lat do CPU}

    {\small
    \begin{labeling}{1822}
    \item[1822] Difference Engine\footnote{{\tiny skonstruowany w londyńskim
        Science Museum w 1991}} -- Charles Babbage
    \item[1837] Analytical Engine -- Charles Babbage hardware\footnote{\tiny
        zbudował prototyp CPU w 1871} \footnote{{\tiny w 1906 jego syn, Henry Babbage,
        zbuduje kompletene CPU}}, a Ada Lovelace software
    \item[1941] Konrad Zuse -- Z3, pierwszy programowalny
        komputer\footnote{{\tiny zniszczony podczas bombardowania Berlina przez
        Aliantów}}
    \item[1944] Harvard Mark I -- drugi programowalny komputer
    \item[1971] Intel 4004 -- 4-bitowy mikroprocesor
    \end{labeling}}

    W 2016 UK pozwoliło sprzedać ARM -- czyli po raz drugi wypuścili z rąk ważny
    kawałek technologii.
\end{frame}

\begin{frame}
    \frametitle{Architektury CPU}

    \begin{labeling}{1837}
        \item[1837] Analytical Engine
        \item[1978] x86
        \item[1985] ARM, MIPS
        \item[1991] PowerPC
        \item[2001] Itanium {\tiny (VLIW; \emph{failed})}
        \item[2003] Mill {\tiny (VLIW; \emph{in development})}
        \item[2010] RISC-V
    \end{labeling}
\end{frame}

\section{Wstęp do komputerów}

\begin{frame}
    \frametitle{von Neumann}

    \begin{enumerate}
        \item CPU
        \item RAM
        \item pamięć masowa
        \item I/O
    \end{enumerate}
\end{frame}

\section{Wstęp do języków programowania}

\begin{frame}
    \frametitle{A co to?}

    Sposób na wyrażenie swoich żądań względem maszyny.

    \vspace{1em}

    Kontrakt z demonem -- spełnia rozkazy \emph{dokładnie tak jak są
    wypowiedziane}, bez oglądania się na \emph{intencje} programisty.
\end{frame}

\begin{frame}
    \frametitle{A po co to komu?}
    {\tt mov eax, 0x2a} \textsubscript{(x86)}\\
    vs\\
    {\tt auto x = int\{42\};} \textsubscript{(C++)}

    \vspace{2em}

    Dużo prostsze wydawanie maszynie skomplikowanych rozkazów, i łatwość
    zrozumienia znaczenia programu.\\
    Automatyczna alokacja rejestrów i pamięci; automatyczne skoki; ergonomiczna
    semantyka.

    \vspace{1em}

    Przenośność (ang. \emph{portability}) programów między platformami.
\end{frame}

\begin{frame}
    \frametitle{Umowny podział}

    \begin{enumerate}
        \item compiled \emph{vs} interpreted (JIT?)
        \item typing: static \emph{vs} dynamic, strong \emph{vs} weak
        \item paradigm\footnote{Seven Languages in Seven Weeks; Bruce A. Tate;
            ISBN-13: 978-1-934356-59-3}: functional \emph{vs} object-oriented
            \emph{vs} structural \emph{vs} prototype-based \emph{vs} ...
        \item rodziny: C-like (pochodne po języku ALGOL), ML-like, Lisp-like
        \item ''toy'' \emph{vs} ''real''
    \end{enumerate}

    \vspace{1em}

    \begin{labeling}{Smalltalk}
        \item[C] compiled, static-weak typing, structural
        \item[C++] compiled, static-strong typing, multiparadigm
        \item[Smalltalk] interpreted, dynamic-strong, object-oriented
        \item[OCaml] compiled, static-strong, functional
        \item[Perl] interpreted, dynamic-weak, multiparadigm
    \end{labeling}
\end{frame}

\section{Składniki języka}

\begin{frame}
    \frametitle{Co jest potrzebne w języku?}

    ?{\tiny \textsubscript{(pytanie do sali)}}
\end{frame}

\begin{frame}
    \frametitle{Z punktu widzenia prostego człowieka}

    \begin{enumerate}
        \item \emph{control flow} -- mechanizmy przepływu kontroli, czyli
            sterowania programem
        \item \emph{data structures} -- reprezentacja struktur danych
        \item \emph{I/O} -- zapis i odczyt danych, czyli sposób na interakcję ze
            światem zewnętrznym
    \end{enumerate}
\end{frame}

\begin{frame}
    \frametitle{Jackson Structured Programming}
    \framesubtitle{Control flow}

    Michael Jackson, 1975; Principles of Program Design

    \vspace{1em}

    \begin{enumerate}
        \item \emph{sequence} -- sekwencjonowanie, czyli ustalenie kolejności
            wykonywania operacji
        \item \emph{selection} (\emph{alternative}) -- wybór (alternatywa),
            czyli decyzja o podjęciu jednej z kilku różnych ścieżek wykonania
        \item \emph{iteration} -- iteracja, czyli powtarzanie tych samych kroków
            $n$ razy
    \end{enumerate}

    \vspace{1em}

    Nadaje się do opisu algorytmów, ale nie za bardzo do czegoś więcej. Często
    tak jest z różnymi modelami -- są wygodne w teorii, ale niezbyt praktyczne.
\end{frame}

\begin{frame}
    \frametitle{Warnier/Orr}
    \framesubtitle{Control flow}

    Jean-Dominique Warnier, 1976; Logical construction of programs\\
    Kenneth Orr, 1977; Structured systems development

    \vspace{1em}

    \begin{enumerate}
        \item \emph{recursion} -- rekurencja, czyli sposób na zagnieżdżone
            wykonywanie operacji
        \item \emph{concurrency} -- współbieżność, czyli sposób na wykonywanie
            kilku operacji ''w tym samym czasie'' (naprzemiennie na jednym
            procesorze, lub równolegle\footnote{ten wariant nazywa się
            \emph{parallelism}, i czasem jest podawany obok współbieżności jako
            coś innego} na wielu)
    \end{enumerate}

    \vspace{1em}

    Rekurencja i współbieżność są nieodłącznymi elementami programów, które
    działają w ''prawdziwym świecie''. Bez nich niemożliwe byłoby interaktywne
    używanie komputerów.
\end{frame}

\begin{frame}
    \frametitle{Michael Scott}
    \framesubtitle{Control flow}

    Michael Lee Scott, 2000; Programming language pragmatics\footnote{ISBN
    1-55860-442-1}

    \vspace{1em}

    \begin{enumerate}
        \item \emph{procedural abstraction} -- zbiór operacji opakowany w sposób
            umożliwiający ich wspólne wywołanie, w skrócie: funkcja
        \item \emph{nondeterminacy} -- niedeterminizm, czyli sposób na
            zapewnienie losowości przy wyborze ściezki wykonania
    \end{enumerate}

    \vspace{1em}

    Funkcje i niedeterminizm zamykają bazowe mechanizmy, które służą kontroli
    przepływu.
\end{frame}

\begin{frame}
    \frametitle{Podsumowanie}
    \framesubtitle{Control flow}

    \begin{enumerate}
        \item \emph{sequence}
        \item \emph{selection}
        \item \emph{iteration}
        \item \emph{recursion}
        \item \emph{concurrency}
        \item \emph{procedural abstraction}
        \item \emph{nondeterminism}
    \end{enumerate}
\end{frame}

\begin{frame}
    \frametitle{Bit, nibble\footnote{połowa bajtu, czyli 4 bity}, byte, word,
        half-word, double-word, quad-word...}
    \framesubtitle{Data structures}

    \epigraph{Na początku było słowo}{J 1,1-3}

    Bit - czyli wartość mogąca przechowywać $0$ lub $1$.

    Podstawową jednostką danych obsługiwanych przez CPU jest ''słowo'' -
    sekwencja bitów o pojedynczego długości rejestru. Dla architektury x86-64
    długość słowa to 64 bity.

    Zapis i odczyt słowa w pamięci zazwyczaj są \emph{operacjami atomowymi} co
    ma znaczenie dla programowania współbieżnego.
\end{frame}

\begin{frame}
    \frametitle{Liczby całkowite, ułamki, wartości logiczne}
    \framesubtitle{Data structures}

    Liczby całkowite - ze znakiem ({\tt signed}), bez znaku ({\tt unsigned}).\\
    Liczby zmiennoprzecinkowe - pojedynczej precyzji, podwójnej precyzji.\\
    Wartości logiczne - prawda, fałsz.
\end{frame}

\begin{frame}
    \frametitle{Znaki, napisy}
    \framesubtitle{Data structures}

    Znaki\footnote{kiedyś były najczęśniej szerokości 1 bajty (ASCII), ale
    obecnie, od upowszechnienia się standardu Unicode, są zazwyczaj zmiennej
    długości (UTF-8)} - reprezentujące pojedynczy glif (literę, znak
    interpunkcyjny, itd.) lub symbol kontrolny.\\
    Napisy - reprezentujące sekwencje znaków (np. {\tt ''Hello, World!''}).
\end{frame}

\begin{frame}
    \frametitle{Lista, kolejka, stos, zbiór, drzewo, krotka...}
    \framesubtitle{Data structures}

    {\scriptsize
    \begin{enumerate}
        \item \emph{list} -- lista, czyli poszeregowana sekwencja wartości typu
            $T$, do których daje dostęp w dowolnej kolejności (ang.
            \emph{random~access})
        \item \emph{queue} -- kolejka, czyli poszeregowana sekwencja wartości
            typu $T$, do których daje dostęp na zasadzie \emph{FIFO}
        \item \emph{stack} -- stos, czyli poszeregowana sekwencja wartości
            typu $T$, do których daje dostęp na zasadzie \emph{LIFO}
        \item \emph{set} -- zbiór, nieposzeregowana kolekcja wartości typu $T$
        \item \emph{tree} -- drzewo, często wykorzystywane do budowania
            ''map'' czyli struktur asocjacyjnych pozwalających na przechowanie
            wartości typu $T$ pod kluczem typu $K$
        \item \emph{tuple} -- krotka, czyli struktura danych zawierająca $n$ pól
            typów $T_0$, $T_1$, ..., $T_n$
    \end{enumerate}}

    Podstawową złożoną strukturą danych jest \emph{tablica}, czyli sekwencja $n$
    elementów typu $T$. Bazując na tablicach bajtów (czyli de facto surowych,
    wydzielonych obszarach pamięci) można zaimplementować wszystkie powyższe
    struktury danych.
\end{frame}

\begin{frame}
    \frametitle{Typy użytkownika}
    \framesubtitle{Data structures}

    \begin{labeling}{\emph{struct}}
        \item[\emph{enum}] wyliczenie, czyli zdefiniowany zbiór wartości, które
            dany typ może przechować (vide \emph{sum~type})
        \item[\emph{struct}] struktura, czyli typ złożony z kilku pól różnych
                typów -- może przechowywać wszystkie kombinacje wartości pól
                (vide \emph{product~type})
    \end{labeling}

    Języki programowania często zapewniają programistom możliwość tworzenia
    własnych typów danych.
\end{frame}

\begin{frame}
    \frametitle{Wskaźniki i dynamiczna alokacja pamięci}
    \framesubtitle{Data structures}

    Implementacja wielu struktur danych (np. list lub napisów o zmiennej
    długości) byłaby niemożliwa bez wskaźników i dynamicznej alokacji pamięci.
\end{frame}

\begin{frame}
    \frametitle{Podsumowanie}
    \framesubtitle{Data structures}

    \begin{enumerate}
        \item słowo (\emph{word}), bajt (\emph{byte}), bit
        \item tablica (\emph{array})
        \item typy użytkownika (\emph{user-defined type} -- \emph{enum},
            \emph{struct})
        \item wskaźnik (\emph{pointer})
        \item dynamiczna alokacja pamięci (\emph{memory allocation})
        \item typy proste \emph{vs} typy złożone
    \end{enumerate}
\end{frame}

\begin{frame}
    \frametitle{Operacje wejścia-wyjścia}
    \framesubtitle{I/O}

    Sposób na interakcję i wymianę danych ze ''światem zewnętrznym'', czyli
    wszystkim tym co dzieje się poza CPU i pamięcią operacyjną (RAM).

    \begin{enumerate}
        \item \emph{I/O port}, \emph{MMU} (ang. \emph{memory management unit})
        \item \emph{file-descriptor}, \emph{socket} (ten sam interfejs dla
            plików i połączeń w sieci; POSIX)
        \item \emph{memory-mapped file}
    \end{enumerate}

    Zapisując i odczytując bajty da się obsłużyć każdy rodzaj urządzenia -
    monitor, klawiaturę, dysk, ramię robota, silnik, itd.
\end{frame}

\section{C++}

\section{Zastrzeżenia}

\end{document}
